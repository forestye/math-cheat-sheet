%!TEX program = xelatex
\documentclass[8pt, landscape, a4paper]{extarticle}

% --- 核心宏包 ---
\usepackage[UTF8, fontset=fandol]{ctex}
\usepackage[margin=0.8cm, top=1cm, bottom=1.3cm]{geometry}
\usepackage{multicol}
\usepackage{xcolor}
\usepackage{tcolorbox}
\usepackage{enumitem}
\usepackage{amsmath}
\usepackage{amssymb}
\usepackage{fontspec}

% --- 去掉页码 ---
\pagestyle{empty}

% --- 颜色定义 ---
\definecolor{headerblue}{RGB}{41, 128, 185}
\definecolor{section1}{RGB}{22, 160, 133} 
\definecolor{section2}{RGB}{41, 128, 185} 
\definecolor{section3}{RGB}{142, 68, 173} 
\definecolor{section4}{RGB}{211, 84, 0}   
\definecolor{section5}{RGB}{192, 57, 43}  
\definecolor{section6}{RGB}{44, 62, 80}   
\definecolor{dividergray}{RGB}{220, 220, 220}

% --- 全局设置 ---
\setlength{\parindent}{0pt}
\setlength{\columnsep}{0.4cm} 
\linespread{1.1} 

% --- 列表样式 ---
\setlist[itemize]{leftmargin=1.5em, nosep, itemsep=2pt, topsep=2pt, label=$\vcenter{\hbox{\tiny$\bullet$}}$}
\setlist[description]{leftmargin=0.2em, style=sameline, nosep, itemsep=2pt, font=\bfseries}

% --- Box 样式 ---
\newtcolorbox{mybox}[2][]{%
  colback=white,
  colframe=#2,
  coltitle=white,
  boxrule=1pt,             
  arc=2mm,                 
  left=5pt, right=5pt, top=5pt, bottom=5pt, 
  toptitle=3pt, bottomtitle=3pt, 
  fonttitle=\bfseries\sffamily\large,
  title={#1},
  after skip=6pt          
}

% --- 自定义命令 ---
\newcommand{\subt}[1]{{\vspace{3pt}\textbf{\large \textcolor{black}{#1}}}}

\newcommand{\boxdesc}[1]{%
    \textit{\small \textcolor{gray}{#1}}%
    \par\vspace{3pt}%
    {\color{dividergray}\hrule height 0.5pt}%
    \vspace{3pt}%
}

\newcommand{\sepline}{%
    \vspace{3pt}%
    {\color{dividergray}\hrulefill}%
    \vspace{1pt}%
}

\begin{document}

% --- 页眉 ---
\begin{center}
    {\Huge \textbf{\sffamily 高阶程序员数学武器库 Cheat Sheet}} \\
    \vspace{0.2cm}
    {\large \texttt{Math for Advanced Developers: From Engineering to Truth}}
\end{center}

% --- 开始四栏布局 ---
\begin{multicols*}{4}

% === 第一栏 ===

\begin{mybox}[�� 极简决策树 (Decision Tree)]{headerblue}
    % 【调整1】增加了 BoxDesc
    \boxdesc{快速定位技术瓶颈的数学解法} 
    
    % 调整了间距以填满空间
    \begin{itemize}[itemsep=4pt] 
        \item 有\textbf{波}或\textbf{周期}? $\to$ 傅里叶 / 复变
        \item 有\textbf{网络}或\textbf{依赖}? $\to$ 图论
        \item 系统\textbf{不稳}或\textbf{发散}? $\to$ 拉普拉斯 / 极点
        \item 数据\textbf{存不下}或\textbf{算太慢}? $\to$ 线代 / 信息论
        \item 资源\textbf{分配}或\textbf{调度}? $\to$ 凸优化 (规划)
        \item 协议\textbf{设计}或\textbf{校验}? $\to$ 抽象代数 / 形式化
        \item 要搞\textbf{加密}或\textbf{纠错}? $\to$ 有限域 (GF)
        \item 要搞\textbf{架构}或\textbf{编译器}? $\to$ 范畴论 / 类型论
    \end{itemize}
\end{mybox}
\vspace*{\fill}
\begin{mybox}[1. 基础与描述 (Foundation)]{section1}
    \boxdesc{描述世界的基石,日常开发最常用}
    
    \subt{线性代数 (Linear Algebra)}
    \begin{itemize}
        \item \textbf{核心}: 空间的变换 (Space Transformation)
        \item \textbf{关键}: 矩阵, 特征值, SVD, 秩
        \item \textbf{应用}: AI 训练, 推荐系统, PCA 降维
    \end{itemize}
    \sepline
    
    \subt{概率统计 (Probability)}
    \begin{itemize}
        \item \textbf{核心}: 处理不确定性
        \item \textbf{关键}: 分布 (高斯/泊松), 贝叶斯 $P(A|B)$
        \item \textbf{应用}: 负载均衡, A/B 测试, 风控
    \end{itemize}
    \sepline
    
    \subt{图论 (Graph Theory)}
    \begin{itemize}
        \item \textbf{核心}: 关系与结构
        \item \textbf{关键}: DAG, 拓扑排序, 最短路径
        \item \textbf{应用}: 依赖管理 (Pip), 死锁检测, 调度
    \end{itemize}
    \sepline
    
    \subt{信息论 (Information Theory)}
    \begin{itemize}
        \item \textbf{核心}: 数据的量化
        \item \textbf{关键}: 熵 (Entropy), 互信息, 交叉熵
        \item \textbf{应用}: 压缩 (Huffman), 密码强度
    \end{itemize}
\end{mybox}
% 【关键修改2】强制换栏,确保 \fill 生效
\columnbreak
% === 第二栏 ===

\begin{mybox}[2. 信号与变换 (Transforms)]{section2}
    \boxdesc{解决“波动”、“周期”与“稳定性”}

    \subt{复变函数 (Complex Analysis)}
    \begin{itemize}
        \item \textbf{核心}: 旋转与降维
        \item \textbf{公式}: $e^{ix} = \cos x + i\sin x$
        \item \textbf{应用}: 交流电, 流体力学 (保角映射)
    \end{itemize}
    \sepline

    \subt{傅里叶变换 (Fourier)}
    \begin{itemize}
        \item \textbf{核心}: 频域视角 (Time vs Freq)
        \item \textbf{关键}: 卷积定理, FFT ($O(N \log N)$)
        \item \textbf{应用}: 音频处理, 图像压缩, 大数乘法
    \end{itemize}
    \sepline

    \subt{拉普拉斯 / Z 变换}
    \begin{itemize}
        \item \textbf{核心}: 系统稳定性 (S域 / Z域)
        \item \textbf{关键}: 极点 (Poles), 收敛域
        \item \textbf{应用}: 控制系统 (PID), 数字滤波 (DSP)
    \end{itemize}
\end{mybox}
\vspace*{\fill}
\begin{mybox}[3. 计算与博弈 (Computation)]{section3}
    \boxdesc{解决“怎么算最快”和“怎么选最优”}

    \subt{数值分析 (Numerical Analysis)}
    \begin{itemize}
        \item \textbf{痛点}: 浮点误差 ($0.1+0.2 \neq 0.3$)
        \item \textbf{解法}: 迭代法, 条件数分析
        \item \textbf{场景}: 物理引擎, 高频交易, 科学计算
    \end{itemize}
    \sepline

    \subt{凸优化 (Convex Optimization)}
    \begin{itemize}
        \item \textbf{痛点}: 在万千可能中找唯一最优
        \item \textbf{解法}: 梯度下降, 拉格朗日对偶
        \item \textbf{场景}: 深度学习训练, 物流调度
    \end{itemize}
    \sepline

    \subt{博弈论 (Game Theory)}
    \begin{itemize}
        \item \textbf{痛点}: 对手也在变,无静态最优
        \item \textbf{解法}: 纳什均衡 (Nash Equilibrium)
        \item \textbf{场景}: 区块链共识, GAN 对抗网络
    \end{itemize}
\end{mybox}
\columnbreak
% === 第三栏 ===

\begin{mybox}[4. 结构与逻辑 (Logic)]{section4}
    \boxdesc{架构师与安全专家的底层内功}

    \subt{抽象代数 (Abstract Algebra)}
    \begin{itemize}
        \item \textbf{本质}: 定义运算规则 (群/环/域)
        \item \textbf{有限域}: $GF(2^8)$ $\to$ 计算机完美算术
        \item \textbf{应用}: AES 加密, RAID 纠错, ECC
    \end{itemize}
    \sepline

    \subt{范畴论 (Category Theory)}
    \begin{itemize}
        \item \textbf{本质}: 组合与解耦 (Morphism)
        \item \textbf{概念}: Functor, Monad (单子)
        \item \textbf{应用}: 函数式编程, 异步模型
    \end{itemize}
    \sepline

    \subt{类型论 (Type Theory)}
    \begin{itemize}
        \item \textbf{本质}: 代码即真理
        \item \textbf{原理}: Curry-Howard 同构
        \item \textbf{应用}: 形式化验证, 编译器 (Rust)
    \end{itemize}
\end{mybox}
\vspace*{\fill}
\begin{mybox}[5. 高维与动态 (Dynamics)]{section5}
    \boxdesc{AI 和前沿物理的核心}

    \subt{流形 (Manifold)}
    \begin{itemize}
        \item \textbf{定义}: 高维空间中卷曲的低维曲面
        \item \textbf{洞察}: 数据是有形状的 (Embedding)
        \item \textbf{应用}: 深度学习原理, 降维
    \end{itemize}
    \sepline

    \subt{随机微积分 (Stochastic Calc)}
    \begin{itemize}
        \item \textbf{公式}: $dX_t = \mu dt + \sigma dW_t$
        \item \textbf{特征}: 确定性趋势 + 随机噪声
        \item \textbf{应用}: 生成式 AI (Diffusion), 定价
    \end{itemize}
    \sepline

    \subt{拓扑学 (Topology)}
    \begin{itemize}
        \item \textbf{关注}: 连通性与洞 (忽略距离)
        \item \textbf{关键}: 同胚, 贝蒂数 (Betti Numbers)
        \item \textbf{应用}: 拓扑数据分析 (TDA), 鲁棒性
    \end{itemize}
\end{mybox}
\columnbreak
% === 第四栏 ===

\begin{mybox}[6. 上帝视角 (The Endgame)]{section6}
    \boxdesc{触碰物理法则与计算边界}
    
    \subt{泛函分析 (Functional)}
    \begin{itemize}[itemsep=3pt]
        \item 无限维线性代数; 希尔伯特空间
        \item $\hookrightarrow$ 量子计算 / 核方法 (SVM)
    \end{itemize}
    \sepline
    
    \subt{辛几何 (Symplectic)}
    \begin{itemize}[itemsep=3pt]
        \item 能量与相空间体积守恒 (哈密顿)
        \item $\hookrightarrow$ 高保真物理仿真 (HMC)
    \end{itemize}
    \sepline

    \subt{代数几何 (Algebraic Geom)}
    \begin{itemize}[itemsep=3pt]
        \item 方程与形状的统一; 概形 (Scheme)
        \item $\hookrightarrow$ 零知识证明 (ZKP), 隐私计算
    \end{itemize}
    \sepline

    \subt{张量分析 (Tensor Calculus)}
    \begin{itemize}[itemsep=3pt]
        \item 坐标无关的物理真理; 协变/逆变
        \item $\hookrightarrow$ 广义相对论 / AI 底层架构
    \end{itemize}
\end{mybox}
\vspace*{\fill}
\begin{mybox}[�� 核心心法 (Philosophy)]{headerblue}
    % 【调整2】将名言移至 BoxDesc
    \boxdesc{“手中无剑 (不背细节),心中有剑 (深知原理)。”}
    
    % 【调整4】使用 \hspace{2em} 实现中文段首缩进
    \hspace{2em}数学不是枯燥的计算题,而是你认知世界的\textbf{显微镜}与\textbf{望远镜}。这个武器库存在的意义,是从工程的“How”上升到数学的“What”:
    
    % 【调整3】增加了内容,扩充了行距,填满右下角
    \begin{itemize}[itemsep=5pt] 
        \item \textbf{建立索引}:遇到瓶颈时,不要暴力试错,先查表寻找是否存在降维打击的数学工具。
        \item \textbf{洞察本质}:代码是逻辑的载体,工程是数学的近似。透过复杂的框架,看到背后不变的数学结构。
        \item \textbf{识别同构}:发现不同系统间(如电路与流体、逻辑与几何)的数学同构性,实现知识迁移。
        \item \textbf{第一原理}:基于公理推演而非经验堆砌,用数学思维构建反脆弱的系统架构。

    \end{itemize}
    
    \vspace{3pt} 
    \centering\textit{\footnotesize 当你凝视深渊(底层原理)时,深渊也在为你提供无尽的算力。}
\end{mybox}

\end{multicols*}

\end{document}