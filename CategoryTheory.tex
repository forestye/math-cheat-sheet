\documentclass[8pt, landscape, a4paper]{extarticle}

% --- 核心宏包 ---
\usepackage[UTF8]{ctex}
\usepackage[margin=0.8cm, top=1cm, bottom=1.3cm]{geometry}
\usepackage{multicol}
\usepackage{xcolor}
\usepackage{tcolorbox}
\usepackage{enumitem}
\usepackage{amsmath}
\usepackage{amssymb}
\usepackage{fontspec}
\usepackage{tikz}
\usepackage{tikz-cd} % 交换图神器
\usetikzlibrary{arrows.meta, shapes}

% --- 去掉页码 ---
\pagestyle{empty}

% --- 颜色定义 (Purple/Pink 主题) ---
\definecolor{headerblue}{RGB}{155, 89, 182}    % Amethyst
\definecolor{navcolor}{RGB}{211, 84, 0}        % 导航橙
\definecolor{intuitioncolor}{RGB}{41, 128, 185}% 直觉蓝
\definecolor{accentcolor}{RGB}{231, 76, 60}    % 强调红
\definecolor{section2}{RGB}{22, 160, 133}      % 绿色
\definecolor{dividergray}{RGB}{220, 220, 220}

% --- 全局设置 ---
\setlength{\parindent}{0pt}
\setlength{\columnsep}{0.4cm} 
\linespread{1.1} 

% --- 列表样式 ---
\setlist[itemize]{leftmargin=1.2em, nosep, itemsep=2pt, topsep=2pt, label=$\textcolor{headerblue}{\vcenter{\hbox{\tiny$\bullet$}}}$ }
\setlist[description]{leftmargin=0.2em, style=sameline, nosep, itemsep=2pt, font=\bfseries}

% --- Box 样式 ---
\newtcolorbox{mybox}[2][]{%
  colback=white,
  colframe=#2,
  coltitle=white,
  boxrule=1pt,             
  arc=2mm,                 
  left=4pt, right=4pt, top=3pt, bottom=3pt, 
  toptitle=3pt, bottomtitle=3pt, 
  fonttitle=\bfseries\sffamily\large,
  title={#1},
  after skip=5pt          
}

% --- 自定义命令 ---
\newcommand{\subt}[1]{{\vspace{2pt}\textbf{\large \textcolor{black}{#1}}}}

\newcommand{\boxdesc}[1]{%
    \textit{\small \textcolor{gray}{#1}}%
    \par\vspace{2pt}%
    {\color{dividergray}\hrule height 0.5pt}%
    \vspace{2pt}%
}

\newcommand{\sepline}{%
    \par \vspace{3pt}%
    {\color{dividergray}\hrule height 0.5pt}%
    \par \vspace{3pt}%
}

% 公式间距
\setlength{\abovedisplayskip}{3pt}
\setlength{\belowdisplayskip}{3pt}

\begin{document}

% --- 页眉 ---
\begin{center}
    {\Huge \textbf{\sffamily \textcolor{headerblue}{范畴论 Category Theory Cheat Sheet}}} \\
    \vspace{0.2cm}
    {\large \texttt{The Mathematics of Mathematics: Patterns, Composition, and Abstraction}}
\end{center}

% --- 开始四栏布局 ---
\begin{multicols*}{4}

% === 第一栏 ===

\begin{mybox}[️ 场景导航 (Use Cases)]{navcolor}
    \boxdesc{遇到什么问题 $\to$ 用什么工具}
    \begin{itemize}[itemsep=2pt]
        \item \textbf{函数式编程 (FP)} $\to$ Monad / Functor
        \item \textbf{程序架构设计} $\to$ 组合性 (Composition)
        \item \textbf{跨领域映射} $\to$ 函子 (Functor)
        \item \textbf{最优解/通用解} $\to$ 泛性质 (Universal Property)
        \item \textbf{逻辑与类型} $\to$ Curry-Howard 同构
        \item \textbf{数据库查询} $\to$ Monad Comprehension
    \end{itemize}
\end{mybox}

\begin{mybox}[1. 基础定义 (The Basics)]{headerblue}
    \boxdesc{对象与态射}
    
    \subt{范畴 (Category) $\mathcal{C}$}
    由以下构成:
    \begin{itemize}
        \item \textbf{Objects ($Ob(\mathcal{C})$)}: 点 (如集合、群、类型)。
        \item \textbf{Morphisms ($Hom(A, B)$)}: 箭头 $f: A \to B$。
    \end{itemize}
    \textbf{公理}:
    \begin{enumerate}
        \item \textbf{组合}: $f \circ g$ 存在。
        \item \textbf{结合律}: $(h \circ g) \circ f = h \circ (g \circ f)$。
        \item \textbf{单位元}: 每个对象有 $id_A$,使得 $f \circ id = f$。
    \end{enumerate}
    \sepline
    
    \subt{常见范畴}
    \begin{itemize}
        \item \textbf{Set}: 集合与函数。
        \item \textbf{Grp}: 群与群同态。
        \item \textbf{Hask}: Haskell 类型与函数。
    \end{itemize}
\end{mybox}

\begin{mybox}[2. 函子 (Functor)]{headerblue}
    \boxdesc{范畴间的映射}
    
    \subt{定义 $F: \mathcal{C} \to \mathcal{D}$}
    \begin{itemize}
        \item 映射对象: $A \mapsto F(A)$。
        \item 映射态射: $f \mapsto F(f)$。
    \end{itemize}
    \textbf{保结构}:
    $$ F(f \circ g) = F(f) \circ F(g), \quad F(id_A) = id_{F(A)} $$
    \textit{编程直觉: \texttt{map} 函数。把函数 $f$ 提升(lift)到容器内部操作。}
    \sepline
    
    \subt{自然变换 (Natural Transformation)}
    函子之间的映射。
    $$ \alpha: F \to G $$
    \textit{编程直觉: 泛型函数。如 \texttt{head :: [a] -> Maybe a}。}
\end{mybox}

\columnbreak

% === 第二栏 ===

\begin{mybox}[3. 单子 (Monad)]{headerblue}
    \boxdesc{自函子的幺半群}
    
    \subt{定义 $(M, \eta, \mu)$}
    \begin{itemize}
        \item $M$: 函子 $M: \mathcal{C} \to \mathcal{C}$。
        \item $\eta$ (Unit): $id \to M$ (把值放入上下文)。
        \item $\mu$ (Join): $M(M(x)) \to M(x)$ (展平上下文)。
    \end{itemize}
    \sepline
    
    \subt{编程中的 Monad}
    处理\textbf{副作用} (Side Effects) 的设计模式。
    \begin{itemize}
        \item \texttt{return}: $\eta$ (Wrap)。
        \item \texttt{bind (>>=)}: 组合带副作用的函数。
    \end{itemize}
    \textbf{常见 Monad}:
    \begin{itemize}
        \item \textbf{Maybe}: 处理空值。
        \item \textbf{List}: 处理非确定性。
        \item \textbf{IO}: 处理输入输出。
        \item \textbf{State}: 处理可变状态。
    \end{itemize}
\end{mybox}

\begin{mybox}[4. 泛性质 (Universal Property)]{headerblue}
    \boxdesc{通过关系定义对象}
    
    \subt{积 (Product) $A \times B$}
    最“好”的包含 $A$ 和 $B$ 的对象。
    任何其他包含 $A, B$ 的对象 $Z$ 都有唯一的映射指向 $A \times B$。
    \textit{编程: Tuple / Struct。}
    \sepline
    
    \subt{上积 (Coproduct) $A + B$}
    最“好”的被 $A$ 和 $B$ 包含的对象。
    \textit{编程: Union / Enum / Sum Type。}
    \sepline
    
    \subt{极限 (Limit) / 上极限 (Colimit)}
    积与上积的推广。
\end{mybox}

\columnbreak

% === 第三栏 ===

\begin{mybox}[5. 伴随 (Adjunction)]{headerblue}
    \boxdesc{范畴论的核心概念}
    
    \subt{定义 $F \dashv G$}
    函子 $F: \mathcal{C} \to \mathcal{D}$ 和 $G: \mathcal{D} \to \mathcal{C}$ 是一对伴随,如果:
    $$ Hom_{\mathcal{D}}(F(A), B) \cong Hom_{\mathcal{C}}(A, G(B)) $$
    \textit{直觉: 最优解与最自由解的对偶。}
    \begin{itemize}
        \item \textbf{左伴随 $F$}: 保留上极限 (Colimits)。
        \item \textbf{右伴随 $G$}: 保留极限 (Limits)。
    \end{itemize}
    \sepline
    
    \subt{例子}
    \begin{itemize}
        \item \textbf{自由群 $\dashv$ 遗忘函子}: 构造最自由的群 vs 忘记群结构只看集合。
        \item \textbf{Currying}: $Hom(A \times B, C) \cong Hom(A, C^B)$。
    \end{itemize}
\end{mybox}

\begin{mybox}[6. 幺半范畴 (Monoidal)]{headerblue}
    \boxdesc{自带乘法的范畴}
    
    \subt{定义 $(\mathcal{C}, \otimes, I)$}
    \begin{itemize}
        \item $\otimes$: 张量积 (Tensor Product)。
        \item $I$: 单位对象。
    \end{itemize}
    \textit{例子: (Set, $\times$, \{*\}) 或 (Vect, $\otimes$, $k$)。}
    \sepline
    
    \subt{应用: 字符串图 (String Diagram)}
    用线代表对象,盒子代表态射。
    量子计算线路图本质上就是幺半范畴的图示。
\end{mybox}

\begin{mybox}[7. Haskell / Scala 实战]{headerblue}
    \boxdesc{代码工具箱}
    \begin{itemize}
        \item \textbf{Functor}: \texttt{fmap (+1) [1,2,3]}
        \item \textbf{Applicative}: \texttt{Just (+1) <*> Just 2}
        \item \textbf{Monad}: \texttt{Just 1 >>= \textbackslash x -> Just (x+1)}
        \item \textbf{Lens}: 也就是 Profunctor Optics。
    \end{itemize}
\end{mybox}

\columnbreak

% === 第四栏 ===

\begin{mybox}[8. 高阶前沿 (Advanced)]{headerblue}
    \boxdesc{深水区}
    
    \subt{Yoneda 引理}
    $$ Hom(Hom(-, A), F) \cong F(A) $$
    \textit{含义: 一个对象完全由它与其他对象的关系决定。}
    \textit{编程: Continuation Passing Style (CPS)。}
    \sepline
    
    \subt{Topos (拓扑斯)}
    像集合范畴一样的范畴。拥有内部逻辑 (Internal Logic)。
    \textit{连接逻辑、几何与代数的桥梁。}
\end{mybox}

\vspace*{\fill}

\begin{mybox}[ 核心直觉 (Intuition)]{intuitioncolor}
    \boxdesc{“结构重于实质。”}
    
    % TikZ 矢量图: 交换图 (Commutative Diagram)
    \begin{center}
    \begin{tikzcd}[row sep=large, column sep=large, ampersand replacement=\&]
        A \arrow[r, "f"] \arrow[d, "h"'] \& B \arrow[d, "g"] \\
        C \arrow[r, "k"] \& D
    \end{tikzcd}
    \end{center}
    \vspace{-10pt}
    \begin{center}
        \small $g \circ f = k \circ h$
    \end{center}

    \hspace{1em}范畴论不关心对象\textbf{内部}是什么 (是数字还是香蕉),它只关心对象之间如何\textbf{连接} (态射)。
    \vspace{4pt}
    
    \subt{三大核心视角}
    \begin{itemize}[itemsep=4pt]
        \item \textbf{组合性 (Compositionality)}: 
        如果一个系统不能被分解为子系统并重新组合,它就是不可理解的。范畴论是关于组合的科学。
        
        \item \textbf{抽象的抽象}: 
        算术抽象了苹果,代数抽象了算术,范畴论抽象了代数。它是数学的高阶语言。
        
        \item \textbf{同一性 (Sameness)}: 
        两个东西什么时候“一样”?
        相等 (Equality) $\to$ 同构 (Isomorphism) $\to$ 等价 (Equivalence)。范畴论提供了更灵活的相等观。
    \end{itemize}
    
    \vspace{6pt}
    \centering\textit{\footnotesize 所有的数学都是范畴论,剩下的只是细节。}
\end{mybox}

\end{multicols*}

\end{document}
